\chapter{環境}
tcolorboxを用いて環境を作成することができます。
\begin{theorem}{ガウス積分}{gauss}
    みんな大好きガウス積分
    \begin{equation}
        \int_{-\infty}^{\infty} e^{-x^2} dx = \sqrt{\pi}
    \end{equation}
    \tcblower
    証明を書くところ
\end{theorem}

\begin{proposition}{とある命題}{some_proposition}
    これは命題です。
\end{proposition}

\begin{corollary}{とある系}{some_corollary}
    これは系です。
    \tcblower
    とある系の証明
\end{corollary}

\begin{lemma}{とある補題}{some_lemma}
    これは補題です。
    \tcblower
    とある補題の証明
\end{lemma}

\begin{definition}{とある定義}{some_definition}
    これは定義です。
    \tcblower
    とある定義の説明であったり例であったり、いろいろ書くところ
\end{definition}

\begin{eg}{とある例}{some_eg}
    これは例題です。
    \tcblower
    例題の解法を書くところ
\end{eg}


\begin{warning}
    これは警告です。最大限の注意を払ってください。
\end{warning}

\begin{question}
    疑問点。わからんこと
\end{question}