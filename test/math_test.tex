\chapter{数式}
\LaTeX では数式を簡単に挿入できます。
\begin{equation}
    \int_{-\infty}^{\infty} e^{-x^2} dx = \sqrt{\pi}
    \label{eq:gaussian}
\end{equation}
また、複数の数式を並べて書くこともできます。
\begin{align}
    \int_{-\infty}^{\infty} e^{-x^2} dx    & = \sqrt{\pi}            \\
    \int_{-\infty}^{\infty}x^2 e^{-x^2} dx & = \frac{\sqrt{\pi}}{2}  \\
    \int_{-\infty}^{\infty}x^4 e^{-x^2} dx & = \frac{3\sqrt{\pi}}{4}
\end{align}

*をつけると番号をつけないこともできます。

\begin{align*}
    \int_{-\infty}^{\infty} e^{-x^2} dx    & = \sqrt{\pi}            \\
    \int_{-\infty}^{\infty}x^2 e^{-x^2} dx & = \frac{\sqrt{\pi}}{2}  \\
    \int_{-\infty}^{\infty}x^4 e^{-x^2} dx & = \frac{3\sqrt{\pi}}{4}
\end{align*}

たくさん並べて一つだけに番号をつけることもできます。
\begin{align*}
    \int_{-\infty}^{\infty} e^{-x^2} dx    & = \sqrt{\pi}                                              \\
    \int_{-\infty}^{\infty}x^2 e^{-x^2} dx & = \frac{\sqrt{\pi}}{2}                                    \\
    \int_{-\infty}^{\infty}x^4 e^{-x^2} dx & = \frac{3\sqrt{\pi}}{4} \numberthis \label{eq:gaussian_4}
\end{align*}

特殊な記号を使うことも可能です。
\begin{equation}
    \N \subset \Z \subset \Q \subset \R \subset \C
\end{equation}

物理関連の記号も使えます。
\begin{equation}
    \Lagr = T - V, \quad \Ham = T + V, \quad\hbar = \frac{h}{2\pi}
\end{equation}

単位については、siコマンドを使うと簡単に書けます。
\begin{equation}
    \SI{1}{\kilogram\meter\per\second\squared} = \SI{1}{\newton}
\end{equation}

\begin{equation}
    \pdv{f}{x}
\end{equation}

\begin{equation}
    \vu{x}
\end{equation}

\begin{equation}
    \bra{a}, \ket{b}, \braket{a}{b}
\end{equation}
cancelを使うと、数式の中の項を打ち消すことができます。
\begin{align*}
    \Sigma_{k = 0}^n \qty(\frac{1}{k} - \frac{1}{k + 1}) & = 1 - \cancel{\frac{1}{2}} + \cancel{\frac{1}{2}} - \cancel{\frac{1}{3}} + \cancel{\frac{1}{3}} - \cancel{\frac{1}{4}} + \cdots \\
                                                         & = 1 - \frac{1}{n + 1} = \frac{n}{n + 1}
\end{align*}

tcolorboxを使うと、数式をハイライトして強調することができます。
\begin{equation}
    \tcbhighmath{
        \int_{-\infty}^{\infty} e^{-x^2} dx = \sqrt{\pi}
    }
\end{equation}