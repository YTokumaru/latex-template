\chapter{環境}

このテンプレートには以下の環境(environment)が\verb|environment.tex|に定義されている。

\begin{proposition}{<命題名>}{<proposition_label>}
    <命題内容>
    \tcblower
    <証明内容(省略可)>
\end{proposition}

\begin{definition}{<定義名>}{<definition_label>}
    <定義内容>
    % \tcblower
    % <証明内容(省略可)>
\end{definition}

\begin{theorem}{<定理名>}{<throrem_label>}
    <定理内容>
    \tcblower
    <証明内容(省略可)>
\end{theorem}

\begin{corollary}{<系名>}{<corollary_label>}
    <系の内容>
    \tcblower
    <証明内容(省略可)>
\end{corollary}

\begin{lemma}{<補題名>}{<lemma_label>}
    <補題の内容>
    \tcblower
    <証明内容(省略可)>
\end{lemma}

\begin{eg}{<例名>}{<eg_label>}
    <例の内容>
    \tcblower
    <解法など(省略可)>
\end{eg}

\begin{problem}{<問題名>}{<problem_label>}
<問題内容>
\tcblower
<解法など(省略可)>
\end{problem}

\begin{warning}
    <警告内容>
\end{warning}

\begin{question}
    <疑問内容>
\end{question}

\section{その他のコマンド}

以下のコマンドを用いると数式を強調することができる。

\begin{longtable}{ccc}
    \toprule
    コマンド               & 出力                              & 説明       \\
    \midrule
    \verb|\highlight{a+b}| & \(\displaystyle \highlight{a+b}\) & 数式の強調 \\
    \bottomrule
\end{longtable}