\chapter{ユーザー定義のコマンド}

このpreambleには以下のユーザー定義のコマンドが\verb|commands.tex|に定義されている。

\section{集合に関連するコマンド}

\begin{longtable}{ccc}
    \toprule
    コマンド    & 出力     & 説明               \\
    \midrule
    \verb|\N|   & \(\N\)   & 自然数             \\
    \verb|\Z|   & \(\Z\)   & 整数               \\
    \verb|\Q|   & \(\Q\)   & 有理数             \\
    \verb|\R|   & \(\R\)   & 実数               \\
    \verb|\C|   & \(\C\)   & 複素数             \\
    \verb|\K|   & \(\K\)   & 上記の体のいずれか \\
    \verb|\Ima| & \(\Ima\) & 像                 \\
    \bottomrule
\end{longtable}

\section{線形代数に関連するコマンド}

\begin{longtable}{ccc}
    \toprule
    コマンド         & 出力          & 説明     \\
    \midrule
    \verb|\identity| & \(\identity\) & 単位行列 \\
    \bottomrule
\end{longtable}

\section{微積分に関連するコマンド}

\begin{longtable}{ccc}
    \toprule
    コマンド             & 出力                            & 説明         \\
    \midrule
    \verb|\vb{v}|        & \(\vb{v}\)                      & ベクトル表記 \\
    \verb|\dv{f}{x}|     & \(\displaystyle \dv{f}{x}\)     & 一階の微分   \\
    \verb|\dv[2]{f}{x}|  & \(\displaystyle \dv[2]{f}{x}\)  & 二階の微分   \\
    \verb|\dv[n]{f}{x}|  & \(\displaystyle \dv[n]{f}{x}\)  & n階の微分    \\
    \verb|\pdv{f}{x}|    & \(\displaystyle \pdv{f}{x}\)    & 一階の偏微分 \\
    \verb|\pdv[2]{f}{x}| & \(\displaystyle \pdv[2]{f}{x}\) & 二階の偏微分 \\
    \verb|\pdv[n]{f}{x}| & \(\displaystyle \pdv[n]{f}{x}\) & n階の偏微分  \\
    \verb|\curl{v}|      & \(\displaystyle \curl{v}\)      & 回転         \\
    \verb|\rot{v}|       & \(\displaystyle \rot{v}\)       & 回転         \\
    \verb|\grad{v}|      & \(\displaystyle \grad{v}\)      & 勾配         \\
    \verb|\diver{v}|     & \(\displaystyle \diver{v}\)     & 発散         \\
    \bottomrule
\end{longtable}

\section{物理に関連するコマンド}

\begin{longtable}{ccc}
    \toprule
    コマンド     & 出力      & 説明           \\
    \midrule
    \verb|\Lagr| & \(\Lagr\) & ラグランジアン \\
    \verb|\Ham|  & \(\Ham\)  & ハミルトニアン \\
    \bottomrule
\end{longtable}

\section{情報理論に関連するコマンド}

\begin{longtable}{ccc}
    \toprule
    コマンド        & 出力         & 説明             \\
    \midrule
    \verb|\binentr| & \(\binentr\) & 二値エントロピー \\
    \bottomrule
\end{longtable}