\chapter{Preambleを読み込むだけでできること}
\section{最初のセクション}
親譲りの無鉄砲で小供の時から損ばかりしている。\footnote{これは小説『坊っちゃん』の冒頭分である。}
小学校に居る時分学校の二階から飛び降りて一週間ほど腰を抜かした事がある。なぜそんな無闇をしたと聞く人があるかも知れぬ。
別段深い理由でもない。新築の二階から首を出していたら、同級生の一人が冗談に、いくら威張っても、そこから飛び降りる事は出来まい。
弱虫やーい。と囃したからである。小使に負ぶさって帰って来た時、おやじが大きな眼をして二階ぐらいから飛び降りて腰を抜かす奴があるかと云ったから、この次は抜かさずに飛んで見せますと答えた。(青空文庫より)
親譲りの無鉄砲で小供の時から損ばかりしている。小学校に居る時分学校の二階から飛び降りて一週間ほど腰を抜かした事がある。なぜそんな無闇をしたと聞く人があるかも知れぬ。別段深い理由でもない。新築の二階から首を出していたら、同級生の一人が冗談に、いくら威張っても、そこから飛び降りる事は出来まい。
\begin{equation}
    \label{eq:faraday}
    \nabla \times \vb{E} = -\pdv{\vb{B}}{t}
\end{equation}
別段深い理由でもない。新築の二階から首を出していたら、同級生の一人が冗談に、いくら威張っても、そこから飛び降りる事は出来まい。弱虫やーい。と囃したからである。小使に負ぶさって帰って来た時、おやじが大きな眼をして二階ぐらいから飛び降りて腰を抜かす奴があるかと云ったから、この次は抜かさずに飛んで見せますと答えた。(青空文庫より)親譲りの無鉄砲で小供の時から損ばかりしている。小学校に居る時分学校の二階から飛び降りて一週間ほど腰を抜かした事がある。なぜそんな無闇をしたと聞く人があるかも知れぬ。別段深い理由でもない。新築の二階から首を出していたら、同級生の一人が冗談に、いくら威張っても、そこから飛び降りる事は出来まい。弱虫やーい。と囃したからである。小使に負ぶさって帰って来た時、おやじが大きな眼をして二階ぐらいから飛び降りて腰を抜かす奴があるかと云ったから、この次は抜かさずに飛んで見せますと答えた。(青空文庫より)親譲りの無鉄砲で小供の時から損ばかりしている。小学校に居る時分学校の二階から飛び降りて一週間ほど腰を抜かした事がある。なぜそんな無闇をしたと聞く人があるかも知れぬ。
\begin{equation}
    \label{eq:refraction}
    \highlight{\frac{\sin \theta_1}{\sin \theta_2} = \frac{v_1}{v_2} = \frac{n_2}{n_1}}
\end{equation}
別段深い理由でもない。新築の二階から首を出していたら、同級生の一人が冗談に、いくら威張っても、そこから飛び降りる事は出来まい。
弱虫やーい。と囃したからである。小使に負ぶさって帰って来た時、おやじが大きな眼をして二階ぐらいから飛び降りて腰を抜かす奴があるかと云ったから、この次は抜かさずに飛んで見せますと答えた。(青空文庫より)親譲りの無鉄砲で小供の時から損ばかりしている。
\begin{figure}[H]
    \centering
    \includesvg[width=0.5\textwidth]{./figures/cylinder.drawio.svg}
    \caption{一般的な円柱}
    \label{fig:cylinder}
\end{figure}
小学校に居る時分学校の二階から飛び降りて一週間ほど腰を抜かした事がある。
なぜそんな無闇をしたと聞く人があるかも知れぬ。
別段深い理由でもない。新築の二階から首を出していたら、同級生の一人が冗談に、いくら威張っても、そこから飛び降りる事は出来まい。弱虫やーい。
と囃したからである。小使に負ぶさって帰って来た時、おやじが大きな眼をして二階ぐらいから飛び降りて腰を抜かす奴があるかと云ったから、この次は抜かさずに飛んで見せますと答えた。(青空文庫より)親譲りの無鉄砲で小供の時から損ばかりしている。小学校に居る時分学校の二階から飛び降りて一週間ほど腰を抜かした事がある。
なぜそんな無闇をしたと聞く人があるかも知れぬ。別段深い理由でもない。新築の二階から首を出していたら、同級生の一人が冗談に、いくら威張っても、そこから飛び降りる事は出来まい。弱虫やーい。
と囃したからである。小使に負ぶさって帰って来た時、おやじが大きな眼をして二階ぐらいから飛び降りて腰を抜かす奴があるかと云ったから、この次は抜かさずに飛んで見せますと答えた。(青空文庫より)親譲りの無鉄砲で小供の時から損ばかりしている。小学校に居る時分学校の二階から飛び降りて一週間ほど腰を抜かした事がある。なぜそんな無闇をしたと聞く人があるかも知れぬ。別段深い理由でもない。

\section{次のセクション}
親譲りの無鉄砲で小供の時から損ばかりしている。小学校に居る時分学校の二階から飛び降りて一週間ほど腰を抜かした事がある。なぜそんな無闇をしたと聞く人があるかも知れぬ。別段深い理由でもない。新築の二階からなぜそんな無闇をしたと聞く人があるかも知れぬ。
別段深い理由でもない。新築の二階から首を出していたら、同級生の一人が冗談に、いくら威張っても、そこから飛び降りる事は出来まい。弱虫やーい。
と囃したからである。小使に負ぶさって帰って来た時、おやじが大きな眼をして二階ぐらいから飛び降りて腰を抜かす奴があるかと云ったから、この次は抜かさずに飛んで見せますと答えた。(青空文庫より)親譲りの無鉄砲で小供の時から損ばかりしている。小学校に居る時分学校の二階から飛び降りて一週間ほど腰を抜かした事がある。なぜそんな無闇をしたと聞く人があるかも知れぬ。別段深い理由でもない。

\begin{figure}[H]
    \centering
    \begin{subfigure}[b]{0.2\textwidth}
        \centering\includegraphics[width=\textwidth]{figures/circle.drawio.png}
        \caption{円}
        \label{fig:circle}
    \end{subfigure}
    \hfill
    \begin{subfigure}[b]{0.2\textwidth}
        \centering\includegraphics[width=\textwidth]{figures/square.drawio.png}
        \caption{正方形}
        \label{fig:square}
    \end{subfigure}
    \hfill
    \begin{subfigure}[b]{0.2\textwidth}
        \centering\includegraphics[width=\textwidth]{figures/triangle.drawio.png}
        \caption{三角形}
        \label{fig:triangle}
    \end{subfigure}
    \caption{様々な図形}
    \label{fig:shapes}
\end{figure}

親譲りの無鉄砲で小供の時から損ばかりしている。小学校に居る時分学校の二階から飛び降りて一週間ほど腰を抜かした事がある。なぜそんな無闇をしたと聞く人があるかも知れぬ。別段深い理由でもない。新築の二階から首を出していたら、同級生の一人が冗談に、いくら威張っても、そこから飛び降りる事は出来まい。弱虫やーい。と囃したからである。小使に負ぶさって帰って来た時、おやじが大きな眼をして二階ぐらいから飛び降りて腰

\begin{theorem}{ピタゴラスの定理}{pythagoras}
    直角三角形において、直角を挟んでいる2辺の長さをそれぞれ\(a, b\)とすると、斜辺の長さ\(c\)は
    \begin{equation}
        c = \sqrt{a^2 + b^2}
    \end{equation}
    となる。
    \tcblower
    4つの直角三角形を並べて下のように正方形を作る。…
\end{theorem}

定理\ref{thm:pythagoras}は三平方の定理とも呼ばれる。

親譲りの無鉄砲で小供の時から損ばかりしている。小学校に居る時分学校の二階から飛び降りて一週間ほど腰を抜かした事がある。なぜそんな無闇をしたと聞く人があるかも知れぬ。別段深い理由でもない。新築の二階から首を出していたら、同級生の一人が冗談に、いくら威張っても、そこから飛び降りる事は出来まい。弱虫やーい。と囃したからである。小使に負ぶさって帰って来た時、おやじが大きな眼をして二階ぐらいから飛び降りて腰

\begin{eg}{一次方程式}{linear_equation}
    次の方程式を解け。
    \begin{enumerate}
        \item \(x + 2 = 0\)
        \item \(2x - 3 = 0\)
        \item \(3x + 4 = 0\)
    \end{enumerate}
    \tcblower
    解のみを記す。
    \begin{enumerate}
        \item \(x = -2\)
        \item \(x = \frac{3}{2}\)
        \item \(x = -\frac{4}{3}\)
    \end{enumerate}
\end{eg}

親譲りの無鉄砲で小供の時から損ばかりしている。小学校に居る時分学校の二階から飛び降りて一週間ほど腰を抜かした事がある。なぜそんな無闇をしたと聞く人があるかも知れぬ。別段深い理由でもない。新築の二階から首を出していたら、同級生の一人が冗談に、いくら威張っても、そこから飛び降りる事は出来まい。弱虫やーい。と囃したからである。小使に負ぶさって帰って来た時、おやじが大きな眼をして二階ぐらいから飛び降りて腰
親譲りの無鉄砲で小供の時から損ばかりしている。小学校に居る時分学校の二階から飛び降りて一週間ほど腰を抜かした事がある。なぜそんな無闇をしたと聞く人があるかも知れぬ。別段深い理由でもない。新築の二階から首を出していたら、同級生の一人が冗談に、いくら威張っても、そこから飛び降りる事は出来まい。弱虫やーい。と囃したからである。小使に負ぶさって帰って来た時、おやじが大きな眼をして二階ぐらいから飛び降りて腰
親譲りの無鉄砲で小供の時から損ばかりしている。小学校に居る時分学校の二階から飛び降りて一週間ほど腰を抜かした事がある。なぜそんな無闇をしたと聞く人があるかも知れぬ。別段深い理由でもない。新築の二階から首を出していたら、同級生の一人が冗談に、いくら威張っても、そこから飛び降りる事は出来まい。弱虫やーい。と囃したからである。小使に負ぶさって帰って来た時、おやじが大きな眼をして二階ぐらいから飛び降りて腰
親譲りの無鉄砲で小供の時から損ばかりしている。小学校に居る時分学校の二階から飛び降りて一週間ほど腰を抜かした事がある。なぜそんな無闇をしたと聞く人があるかも知れぬ。別段深い理由でもない。新築の二階から首を出していたら、同級生の一人が冗談に、いくら威張っても、そこから飛び降りる事は出来まい。弱虫やーい。と囃したからである。小使に負ぶさって帰って来た時、おやじが大きな眼をして二階ぐらいから飛び降りて腰
親譲りの無鉄砲で小供の時から損ばかりしている。小学校に居る時分学校の二階から飛び降りて一週間ほど腰を抜かした事がある。なぜそんな無闇をしたと聞く人があるかも知れぬ。別段深い理由でもない。新築の二階から首を出していたら、同級生の一人が冗談に、いくら威張っても、そこから飛び降りる事は出来まい。弱虫やーい。と囃したからである。小使に負ぶさって帰って来た時、おやじが大きな眼をして二階ぐらいから飛び降りて腰

個人的には、グラフはgnuplotやpgfplotsで作成するよりも、pythonのmatplotlibで作成する方が楽であると思う。
なんといっても、データ解析などの結果をそのままグラフにできるのが強みである。
\begin{figure}[H]
    \centering
    \includesvg[width=0.8\textwidth]{figures/sine.svg}
    \caption{サインカーブ}
    \label{fig:plot}
\end{figure}

\begin{warning}
    これは警告です。最大限の注意を払ってください。
\end{warning}

\begin{question}
    疑問点。なぜこのような結果が得られたのか?
\end{question}

親譲りの無鉄砲で小供の時から損ばかりしている。小学校に居る時分学校の二階から飛び降りて一週間ほど腰を抜かした事がある。なぜそんな無闇をしたと聞く人があるかも知れぬ。別段深い理由でもない。新築の二階から首を出していたら、同級生の一人が冗談に、いくら威張っても、そこから飛び降りる事は出来まい。弱虫やーい。と囃したからである。小使に負ぶさって帰って来た時、おやじが大きな眼をして二階ぐらいから飛び降りて腰
親譲りの無鉄砲で小供の時から損ばかりしている。小学校に居る時分学校の二階から飛び降りて一週間ほど腰を抜かした事がある。なぜそんな無闇をしたと聞く人があるかも知れぬ。別段深い理由でもない。新築の二階から

親譲りの無鉄砲で小供の時から損ばかりしている。小学校に居る時分学校の二階から飛び降りて一週間ほど腰を抜かした事がある。なぜそんな無闇をしたと聞く人があるかも知れぬ。別段深い理由でもない。新築の二階から首を出していたら、同級生の一人が冗談に、いくら威張っても、そこから飛び降りる事は出来まい。弱虫やーい。と囃したからである。小使に負ぶさって帰って来た時、おやじが大きな眼をして二階ぐらいから飛び降りて腰
親譲りの無鉄砲で小供の時から損ばかりしている。小学校に居る時分学校の二階から飛び降りて一週間ほど腰を抜かした事がある。なぜそんな無闇をしたと聞く人があるかも知れぬ。別段深い理由でもない。新築の二階から

親譲りの無鉄砲で小供の時から損ばかりしている。小学校に居る時分学校の二階から飛び降りて一週間ほど腰を抜かした事がある。なぜそんな無闇をしたと聞く人があるかも知れぬ。別段深い理由でもない。新築の二階から首を出していたら、同級生の一人が冗談に、いくら威張っても、そこから飛び降りる事は出来まい。弱虫やーい。と囃したからである。小使に負ぶさって帰って来た時、おやじが大きな眼をして二階ぐらいから飛び降りて腰
親譲りの無鉄砲で小供の時から損ばかりしている。小学校に居る時分学校の二階から飛び降りて一週間ほど腰を抜かした事がある。なぜそんな無闇をしたと聞く人があるかも知れぬ。別段深い理由でもない。新築の二階から

\begin{table}[H]
    \begin{subtable}[h]{0.45\textwidth}
        \centering
        \begin{tabular}{l  l  l}
            \toprule
            Day & Max Temp & Min Temp \\
            \midrule
            Mon & 20       & 13       \\
            Tue & 22       & 14       \\
            Wed & 23       & 12       \\
            Thu & 25       & 13       \\
            Fri & 18       & 7        \\
            Sat & 15       & 13       \\
            Sun & 20       & 13       \\
            \bottomrule
        \end{tabular}
        \caption{First Week}
        \label{tab:week1}
    \end{subtable}
    \hfill
    \begin{subtable}[h]{0.45\textwidth}
        \centering
        \begin{tabular}{l  l l}
            \toprule
            Day & Max Temp & Min Temp \\
            \midrule
            Mon & 17       & 11       \\
            Tue & 16       & 10       \\
            Wed & 14       & 8        \\
            Thu & 12       & 5        \\
            Fri & 15       & 7        \\
            Sat & 16       & 12       \\
            Sun & 15       & 9        \\
            \bottomrule
        \end{tabular}
        \caption{Second Week}
        \label{tab:week2}
    \end{subtable}
    \caption{Max and min temps recorded in the first two weeks of July}
    \label{tab:temps}
\end{table}

親譲りの無鉄砲で小供の時から損ばかりしている。小学校に居る時分学校の二階から飛び降りて一週間ほど腰を抜かした事がある。なぜそんな無闇をしたと聞く人があるかも知れぬ。別段深い理由でもない。新築の二階から

\begin{corollary}{とある系}{some_corollary}
    これは系である。
\end{corollary}

\begin{definition}{とある定義}{some_definition}
    これは定義である。
\end{definition}
親譲りの無鉄砲で小供の時から損ばかりしている。小学校に居る時分学校の二階から飛び降りて一週間ほど腰を抜かした事がある。なぜそんな無闇をしたと聞く人があるかも知れぬ。別段深い理由でもない。新築の二階から
親譲りの無鉄砲で小供の時から損ばかりしている。小学校に居る時分学校の二階から飛び降りて一週間ほど腰を抜かした事がある。なぜそんな無闇をしたと聞く人があるかも知れぬ。別段深い理由でもない。新築の二階から
親譲りの無鉄砲で小供の時から損ばかりしている。小学校に居る時分学校の二階から飛び降りて一週間ほど腰を抜かした事がある。なぜそんな無闇をしたと聞く人があるかも知れぬ。別段深い理由でもない。新築の二階から

\begin{lemma}{とある補題}{some_lemma}
    これは補題である。
\end{lemma}