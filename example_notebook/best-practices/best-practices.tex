\chapter{ベストプラクティス}
\label{chap:best-practices}

\section{\LaTeX を書く環境}

\LaTeX を書くための環境を整えることで、効率的にレポートを書くことができる。

\subsection{\LaTeX を書ける主な環境}
\subsubsection{オンライン環境}
近年はオンラインで\LaTeX を書くことができる環境も増えてきている。
手っ取り早く\LaTeX で書き始めることができ、環境構築の負担はほとんどない。
\LaTeX を使い始める初心者にはオススメの方法である。

一方、オンラインでの環境はインターネットへの接続を前提としているため、出先で作業をするといったことに向いていないというデメリットもある。
また、後述するデータ解析からレポート執筆までのフローを自動化するためには、ローカル環境での作業が必要となる。

\subsubsection{Windows環境}
Windowsに\TeX Liveをインストールすることで、ローカル環境で\LaTeX を書くことができるようになる。
しかし、\LaTeX 周りのツールをインストールするのが面倒というデメリットもある。
\verb|choco|などのパッケージマネージャーを使ってツールをインストールし、環境変数を適切に設定する必要がある。
しかし、Windows上での\LaTeX 環境構築についての文献は少ないためあまりおすすめできない。
WindowsユーザーはWSL2などのLinuxの仮想環境上で\LaTeX を書くことを勧める。

\subsubsection{Linux環境}
最も楽に\LaTeX の環境を構築できるのはLinux環境である。
これはインターネット上に多くの文献(日英ともに)が存在し、トラブルシューティングがしやすいためである。
\LaTeX のコンパイル時に必要となるプログラムもデフォルトのパッケージマネージャーでインストールできる場合がほとんどなので、文献上のコマンドをそのまま実行するだけで環境構築を進める事ができる。

