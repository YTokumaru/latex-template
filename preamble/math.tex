\usepackage{amsmath}        % alignその他
\usepackage{amssymb}        % \thereforeなど

% *のついた環境内の数式に番号をつける
\newcommand{\numberthis}{\addtocounter{equation}{1}\tag{\theequation}}

\usepackage{amsfonts}    % \mathbb{R}などの数式用フォント

% 集合関連の記号
\newcommand{\N}{\ensuremath{\mathbb{N}}}
\newcommand{\Z}{\ensuremath{\mathbb{Z}}}
\newcommand{\Q}{\ensuremath{\mathbb{Q}}}
\newcommand{\R}{\ensuremath{\mathbb{R}}}
\newcommand{\C}{\ensuremath{\mathbb{C}}}

\usepackage{physics}        % 物理関連の数式用

% 物理関連の記号
\newcommand{\Lagr}{\ensuremath{\mathcal{L}}}     % Lagrangian
\newcommand{\Ham}{\ensuremath{\mathcal{H}}}      % Hamiltonian

\usepackage[per-mode=symbol]{siunitx}           % 単位。デフォルトでper-mode=symbol:分母の表記を/にする

\usepackage{cancel}             % 項の打ち消しの線を引く

\usepackage[many]{tcolorbox}      % 数式をハイライトして強調する
\newcommand{\highlight}[1]{\ensuremath\tcbhighmath{#1}}